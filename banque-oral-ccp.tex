\documentclass[a4paper,12pt,oneside]{book}

% =============================================================================
% ENCODING AND LANGUAGE
% =============================================================================
\usepackage[utf8]{inputenc}
\usepackage[T1]{fontenc}
\usepackage[french]{babel}

% =============================================================================
% FONTS AND TYPOGRAPHY
% =============================================================================
\usepackage{lmodern}
\usepackage{microtype}

% =============================================================================
% PAGE LAYOUT AND GEOMETRY
% =============================================================================
\usepackage{geometry}
\geometry{
	top=2.5cm,
	bottom=2.5cm,
	left=2.5cm,
	right=2.5cm,
	headheight=15pt
}

% =============================================================================
% GRAPHICS AND COLORS
% =============================================================================
\usepackage{graphicx}
\usepackage{xcolor}
\newcommand{\hl}[1]{%
	\colorbox{yellow!50}{#1}%
}
% =============================================================================
% MATHEMATICS
% =============================================================================
\usepackage{amsmath, amssymb}
\usepackage{mathtools}
\usepackage{stmaryrd}

% =============================================================================
% BOXES AND ENVIRONMENTS
% =============================================================================
\usepackage[most]{tcolorbox}
\tcbuselibrary{breakable,skins}

% =============================================================================
% HEADERS AND FOOTERS
% =============================================================================
\usepackage{fancyhdr}

% =============================================================================
% HYPERLINKS
% =============================================================================
\usepackage{hyperref}
\hypersetup{
	colorlinks=true,
	linkcolor=blue,
	urlcolor=blue,
	citecolor=blue,
	bookmarksdepth=3,
	pdfhighlight=/I,
	pdftitle={Gaya Math},
	pdfauthor={SoG},
	pdfsubject={Mathématiques},
	pdfkeywords={mathématiques, algèbre, analyse}
}

% =============================================================================
% SYMBOLS AND DECORATIONS
% =============================================================================
\usepackage{pifont}

% =============================================================================
% CUSTOM MATHEMATICAL COMMANDS
% =============================================================================

% Fields and number sets
\newcommand{\K}{\mathbb{K}}
\newcommand{\C}{\mathbb{C}}
\newcommand{\R}{\mathbb{R}}
\newcommand{\N}{\mathbb{N}}
\newcommand{\Z}{\mathbb{Z}}
\newcommand{\Q}{\mathbb{Q}}

% Linear algebra
\newcommand{\GL}{\mathrm{GL}}
\newcommand{\Id}{\mathrm{Id}}
\newcommand{\Ker}{\mathrm{Ker}}
\newcommand{\Imf}{\mathrm{Im}}
\newcommand{\Com}{\mathrm{Com}}
\newcommand{\Sp}{\mathrm{Sp}}
\newcommand{\Vect}{\mathrm{Vect}}
\newcommand{\tr}{\mathrm{tr}}
\newcommand{\rg}{\mathrm{rg}}

% Number theory
\newcommand{\pgcd}{\mathrm{pgcd}}
\newcommand{\ppcm}{\mathrm{ppcm}}

% Matrix spaces
\newcommand{\Mn}{\mathcal{M}_n(\K)}
\newcommand{\MnC}{\mathcal{M}_n(\C)}

% Characteristic polynomials
\newcommand{\chiA}{\chi_A}
\newcommand{\muA}{\mu_A}
\newcommand{\PiA}{\Pi_A}

% Proof end symbol
\newcommand{\qed}{\hfill$\square$}

% =============================================================================
% PAGE STYLE CONFIGURATION
% =============================================================================
\pagestyle{fancy}
\fancyhf{}
\fancyhead[L]{\nouppercase{\leftmark}}
\fancyhead[R]{\thepage}
\fancyfoot[C]{\hyperref[sec:sommaire]{\small Retour au sommaire}}
\renewcommand{\headrulewidth}{0.4pt}
\renewcommand{\footrulewidth}{0.4pt}

% =============================================================================
% TCOLORBOX STYLES
% =============================================================================

% Base style for all boxes
\tcbset{
	basestyle/.style={
		enhanced,
		breakable,
		arc=3pt,
		outer arc=0pt,
		left=12pt,
		right=12pt,
		top=8pt,
		bottom=8pt,
		boxsep=5pt,
		width=\linewidth,
		halign=justify,
		fonttitle=\bfseries,
		before skip=12pt,
		after skip=12pt,
	}
}

% Statement box
\newtcolorbox{statementbox}[1][]{%
	basestyle,
	colback=gray!8,
	colframe=gray!60!black,
	title={\textbf{Énoncé}},
	title after break={\textbf{Énoncé (suite)}},
	#1
}

% Solution box
\newtcolorbox{solutionbox}[1][]{%
	basestyle,
	colback=blue!8,
	colframe=blue!60!black,
	title={\textbf{Solution}},
	title after break={\textbf{Solution (suite)}},
	#1
}

% Technique box
\newtcolorbox{techniquebox}[1][]{%
	basestyle,
	colback=yellow!12,
	colframe=orange!70!black,
	title={\textbf{Technique à mémoriser}},
	title after break={\textbf{Technique à mémoriser (suite)}},
	#1
}

% Formula box
\newtcolorbox{formbox}[1][]{%
	basestyle,
	colback=green!12,
	colframe=green!60!black,
	title={\textbf{Formulaire}},
	title after break={\textbf{Formulaire (suite)}},
	#1
}

% =============================================================================
% SIMPLIFIED ENVIRONMENTS
% =============================================================================
\newenvironment{statement}{\begin{statementbox}}{\end{statementbox}}
\newenvironment{solution}{\begin{solutionbox}}{\end{solutionbox}}
\newenvironment{technique}{\begin{techniquebox}}{\end{techniquebox}}
\newenvironment{form}{\begin{formbox}}{\end{formbox}}

% =============================================================================
% DOCUMENT CONTENT
% =============================================================================
\begin{document}
	
	% Document metadata
	\author{SoG}
	\title{Gaya Math}
	\date{Juillet 2025}
	
	% Front matter
	\frontmatter
	\maketitle
	
	% Table of contents configuration
	\setcounter{tocdepth}{3}
	\setcounter{secnumdepth}{3}
	\tableofcontents
	\label{sec:sommaire}
	
	% Main content
	\mainmatter
	
\section*{exercise[59]}
\begin{statement}
	Soit $n$ un entier naturel tel que $n \geq 2$.\\
	Soit $E$ l’espace vectoriel des polynômes à coefficients dans $\mathbb{K}$ ($\mathbb{K} = \mathbb{R}$ ou $\mathbb{K} = \mathbb{C}$) de degré inférieur ou égal à $n$.\\
	On pose : $\forall P \in E, \ f(P) = P - P'$.
	\begin{enumerate}
		\item Démontrer que $f$ est bijectif de deux manières :
		\begin{itemize}
			\item[(a)] sans utiliser de matrice de $f$,
			\item[(b)] en utilisant une matrice de $f$.
		\end{itemize}
		\item Soit $Q \in E$. Trouver $P$ tel que $f(P) = Q$.\\
		\textit{Indication :} si $P \in E$, quel est le polynôme $P^{(n+1)}$ ?
		\item $f$ est-il diagonalisable ?
	\end{enumerate}
\end{statement}

\begin{solution}
	\begin{enumerate}
		\item $f(E) \subset E$ :\\
		$\forall P \in E \setminus \{0\}$, $\deg(P - P') = $?\\
		\begin{itemize}
			\item[(a)] $\ker f$.\\
			Si $P \in \ker f$ alors $P - P' = 0$. $\deg(P - P') = $? 
			\item[(b)] La matrice de $f$ dans $e$ est :
			\[
			A = \begin{pmatrix}
				1 & -1 &        &        & 0 \\
				& 1  & \ddots &        &   \\
				&    & \ddots & -n     &   \\
				0  &   &        & 1
			\end{pmatrix}
			\]
		\end{itemize}
		
		\item Soit $P$ tel que $f(P) = Q$. \\
		écrire $P = Q + P'$ et calculer $P^{(n+1)}$
		
		\item Quelles sont les valeurs propres de $f$ ? Qu'impliquerait la diagonalisabilité ?
	\end{enumerate}
\end{solution}


\section{exercise[60]}
	\begin{statement}
		Soit la matrice $A = \begin{pmatrix} 1 & 2 \\ 2 & 4 \end{pmatrix}$ et $f$ l’endomorphisme de $\mathcal{M}_2(\mathbb{R})$ défini par : $f(M) = AM$.
		\begin{enumerate}
			\item Déterminer une base de $\ker f$.
			\item $f$ est-il surjectif ?
			\item Déterminer une base de $\mathrm{Im} f$.
			\item A-t-on $\mathcal{M}_2(\mathbb{R}) = \ker f \oplus \mathrm{Im} f$ ?
		\end{enumerate}
	\end{statement}
	
	\begin{solution}
		\begin{enumerate}
			\item Posons $M = \begin{pmatrix} a & b \\ c & d \end{pmatrix} \in \mathcal{M}_2(\mathbb{R})$.\\
			$f(M) = 0 \Rightarrow M = \begin{pmatrix} -2c & -2d \\ c & d \end{pmatrix}$
			\item $\ker f \ne \{0\} \Rightarrow f$ non injectif.\\
			$f$ est un endomorphisme d’un espace de dimension finie donc $f$ non surjectif.
			\item la formule du rang, comment sont les colonnes de la matrice ?
			\item Décomposer $M$ dans la base de \(\ker f\) et de \(\mathrm{Im}\)
		\end{enumerate}
	\end{solution}

\section*{Exercice 65}

\begin{statement}
	Soit $u$ un endomorphisme d’un espace vectoriel $E$ sur le corps $\mathbb{K}$ ($\mathbb{K} = \mathbb{R}$ ou $\mathbb{K} = \mathbb{C}$). On note $\mathbb{K}[X]$ l’ensemble des polynômes à coefficients dans $\mathbb{K}$.
	\begin{enumerate}
		\item Démontrer que : $\forall (P,Q) \in \mathbb{K}[X] \times \mathbb{K}[X],\ (PQ)(u) = P(u) \circ Q(u)$.
		\item 
		\begin{itemize}
			\item[(a)] Démontrer que : $\forall (P,Q) \in \mathbb{K}[X] \times \mathbb{K}[X],\ P(u) \circ Q(u) = Q(u) \circ P(u)$.
			\item[(b)] Démontrer que, pour tout $(P,Q) \in \mathbb{K}[X] \times \mathbb{K}[X]$ :\\
			\hspace*{1em} $(P$ polynôme annulateur de $u) \Rightarrow (PQ$ polynôme annulateur de $u)$
		\end{itemize}
		\item Soit $A = \begin{pmatrix} 1 & -1 \\ 2 & 2 \end{pmatrix}$.\\
		Écrire le polynôme caractéristique de $A$, puis en déduire que le polynôme $R = X^4 + 2X^3 + X^2 - 4X$ est un polynôme annulateur de $A$.
	\end{enumerate}
\end{statement}

\begin{solution}
		\[
		P_A(X) = \det \begin{pmatrix} X - 1 & 1 \\ -2 & X - 2 \end{pmatrix}
		= (X - 1)(X - 2) + 2 = X^2 - 3X + 4.
		\]
\end{solution}

\section*{Exercice 67}

\begin{statement}
	Soit la matrice
	\[
	M = \begin{pmatrix}
		0 & a & c \\
		b & 0 & c \\
		b & -a & 0
	\end{pmatrix}
	\]
	où $a,b,c$ sont des réels.\\
	$M$ est-elle diagonalisable dans $\mathcal M_3(\mathbb R)$ ? $M$ est-elle diagonalisable dans $\mathcal M_3(\mathbb C)$ ?
\end{statement}

\begin{solution}
	\[
	\chi_M(\lambda) = \lambda \left( \lambda^2 + ca - ba - bc \right)
	\]
	
	\begin{itemize}
		\item \textbf{Premier cas :} $ca - ba - bc < 0$.
		\item \textbf{Deuxième cas :} $ca - ba - bc = 0$.
		\item \textbf{Troisième cas :} $ca - ba - bc > 0$.
	\end{itemize}
\end{solution}

\section*{Exercice 68}

\begin{statement}
	Soit la matrice 
	\[
	A = \begin{pmatrix}
		1 & -1 & 1 \\
		-1 & 1 & -1 \\
		1 & -1 & 1
	\end{pmatrix}.
	\]
	\begin{enumerate}
		\item Montrer que \(A\) est diagonalisable de quatre manières :
		\begin{enumerate}
			\item sans calcul,
			\item en calculant directement le déterminant \(\det(\lambda I_3 - A)\) et en déterminant les sous-espaces propres,
			\item en utilisant le rang de la matrice,
			\item en calculant \(A^2\).
		\end{enumerate}
		\item On note \(f\) l’endomorphisme de \(\mathbb{R}^3\) dont la matrice dans la base canonique est \(A\).
		Trouver une base orthonormée dans laquelle la matrice de \(f\) est diagonale.
	\end{enumerate}
\end{statement}

\begin{solution}
	\begin{enumerate}
		\item 
		\begin{enumerate}
			\item 
			\[
			\chi_A(\lambda) = \det(\lambda I_3 - A) = \lambda^2 (\lambda - 3).
			\]
			\[
			E_3(A) = \mathrm{Vect}\left(\begin{pmatrix}1 \\ -1 \\ 1\end{pmatrix}\right), \quad E_0(A) = \{(x,y,z) \in \mathbb{R}^3 : x - y + z = 0\}.
			\]
		
			\item \(\dim E_0(A) = ?\).
			
			\item Calcul de \(A^2\) :
			\[
			A^2 = 3A,
			\]
			
		\end{enumerate}
		
		\item Une base orthonormée de \(E_3(f)\) est 
		\[
		u = \frac{1}{\sqrt{3}}(1,-1,1).
		\]
		
		Deux vecteurs orthogonaux de \(E_0(f)\) sont 
		\[
		(1,1,0), \quad (1,-1,-2).
		\]
		
		En les normalisant, on pose 
		\[
		v = \frac{1}{\sqrt{2}}(1,1,0), \quad w = \frac{1}{\sqrt{6}}(1,-1,-2).
		\]
		
		
	\end{enumerate}
\end{solution}

\section*{Exercice 69}

\begin{statement}
	On considère la matrice 
	\[
	A = \begin{pmatrix}
		0 & a & 1 \\
		a & 0 & 1 \\
		a & 1 & 0
	\end{pmatrix}
	\]
	où \(a\) est un réel.
	\begin{enumerate}
		\item Déterminer le rang de \(A\).
		\item Pour quelles valeurs de \(a\), la matrice \(A\) est-elle diagonalisable ?
	\end{enumerate}
\end{statement}

\begin{solution}
	\begin{enumerate}
		\item Calcul du rang de \( A \).
		
		\[
		\det A = a(a+1).
		\]
		
		\textbf{Premier cas :} \(a \neq 0\) et \(a \neq -1\)\\
		Alors \(\det A \neq 0\) donc \(A\) est inversible.\\
		Donc \(\mathrm{rg}(A) = 3\).
		
		\textbf{Deuxième cas :} \(a = 0\)
		\[
		A = \begin{pmatrix}
			0 & 0 & 1 \\
			0 & 0 & 1 \\
			0 & 1 & 0
		\end{pmatrix}
		\]
		On remarque que les deux premières lignes sont colinéaires, donc \(\mathrm{rg}(A) = 2\).
		
		\textbf{Troisième cas :} \(a = -1\)
		\[
		A = \begin{pmatrix}
			0 & -1 & 1 \\
			-1 & 0 & 1 \\
			-1 & 1 & 0
		\end{pmatrix}
		\]
		Les deux premières colonnes de \(A\) sont non colinéaires, donc \(\mathrm{rg}(A) \geq 2\).\\
		En calculant le déterminant : \(\det A = a(a+1) = (-1)(0) = 0\), donc \(A\) n'est pas inversible.\\
		Donc \(\mathrm{rg}(A) = 2\).
		
		\item Étude de la diagonalisabilité de \( A \)
		
		On note \(\chi_A(X)\) le polynôme caractéristique de \(A\). On a :
		\[
		\chi_A(X) = (X - a - 1)(X + a)(X + 1).
		\]
		
		Les racines sont donc \(a+1\), \(-a\), et \(-1\).\\
		On étudie les cas où ces racines sont égales ou distinctes :
		
		\begin{itemize}
			\item \textbf{Premier cas :} \(a \neq 1\), \(a \neq -2\) et \(a \neq -\frac{1}{2}\).\\
			Les trois racines sont distinctes, donc \(A\) possède trois valeurs propres distinctes.\\
			\(\Rightarrow A\) est diagonalisable.
			
			\item \textbf{Deuxième cas :} \(a = 1\).\\
			Alors \(\chi_A(X) = (X - 2)(X + 1)^2\).\\
			\(A\) est diagonalisable si et seulement si \(\dim E_{-1} = 2\), i.e. \(\mathrm{rg}(A + I_3) = 1\).
			
			\[
			A + I_3 = \begin{pmatrix}
				1 & 1 & -1 \\
				1 & 1 & -1 \\
				1 & 1 & -1
			\end{pmatrix}
			\Rightarrow \mathrm{rg}(A + I_3) = 1.
			\]
			Donc \(\dim E_{-1} = 2\), \(A\) est diagonalisable.
			
			\item \textbf{Troisième cas :} \(a = -2\).\\
			Alors \(\chi_A(X) = (X + 1)^2 (X - 2)\).\\
			\[
			A + I_3 = \begin{pmatrix}
				1 & 2 & -1 \\
				2 & 1 & -1 \\
				2 & 1 & -1
			\end{pmatrix}
			\Rightarrow \mathrm{rg}(A + I_3) = 2.
			\]
			Donc \(\dim E_{-1} = 1 < 2\).\\
			Or \(-1\) est de multiplicité 2 dans \(\chi_A\).\\
			\(\Rightarrow A\) n’est pas diagonalisable.
			
			\item \textbf{Quatrième cas :} \(a = -\frac{1}{2}\).\\
			Alors \(\chi_A(X) = \left(X + \frac{1}{2}\right)^2(X + 1)\).\\
			\[
			A + \frac{1}{2} I_3 = \begin{pmatrix}
				\frac{1}{2} & -\frac{1}{2} & -1 \\
				-\frac{1}{2} & \frac{1}{2} & -1 \\
				-\frac{1}{2} & 1 & \frac{1}{2}
			\end{pmatrix}
			\Rightarrow \mathrm{rg}\left(A + \frac{1}{2} I_3\right) = 2.
			\]
			Donc \(\dim E_{-\frac{1}{2}} = 1 < 2\), et \(-\frac{1}{2}\) est de multiplicité 2 dans \(\chi_A\).\\
			\(\Rightarrow A\) n’est pas diagonalisable.
		\end{itemize}
	\end{enumerate}
\end{solution}


	\section*{Exercice 70}
	
	\begin{statement}
		Soit 
		\[
		A = \begin{pmatrix}
			0 & 0 & 1 \\
			1 & 0 & 0 \\
			0 & 1 & 0
		\end{pmatrix} \in \mathcal{M}_3(\mathbb{C}).
		\]
		\begin{enumerate}
			\item Déterminer les valeurs propres et les vecteurs propres de \(A\). \(A\) est-elle diagonalisable ?
			\item Soit \((a, b, c) \in \mathbb{C}^3\) et \(B = aI_3 + bA + cA^2\), où \(I_3\) désigne la matrice identité d’ordre 3.\\
			Déduire de la question 1. les éléments propres de \(B\).
		\end{enumerate}
	\end{statement}
	
	\begin{solution}
		\begin{enumerate}
			\item \[
			\chi_A(X) = X^3 - 1 \quad \Rightarrow \quad \mathrm{Sp}(A) = \{1, j, j^2\},
			\]
			\[
			E_1(A) = \ker(A - I_3) = \mathrm{Vect}\left(\begin{pmatrix}1 \\ 1 \\ 1\end{pmatrix}\right), \quad
			\]
			\[E_j(A) = \ker(A - jI_3) = \mathrm{Vect}\left(\begin{pmatrix}1 \\ j^2 \\ j\end{pmatrix}\right), \quad
			\]
			\[E_{j^2}(A) = \ker(A - j^2 I_3) = \mathrm{Vect}\left(\begin{pmatrix}1 \\ j \\ j^2\end{pmatrix}\right).
			\]
			
			\item 
			\[
			P = \begin{pmatrix}
				1 & 1 & 1 \\
				1 & j^2 & j \\
				1 & j & j^2
			\end{pmatrix}, \quad
			D = \begin{pmatrix}
				1 & 0 & 0 \\
				0 & j & 0 \\
				0 & 0 & j^2
			\end{pmatrix}.
			\]
			
			On a alors :
			\[
			A = PDP^{-1} \quad \Rightarrow \quad B = aI_3 + bA + cA^2 = P(aI_3 + bD + cD^2)P^{-1}.
			\]
			\[
			  = P \cdot \mathrm{diag}(Q(1), Q(j), Q(j^2)) \cdot P^{-1}.
			\]
			
			\textbf{Premier cas :} les valeurs \(Q(1), Q(j), Q(j^2)\) sont toutes distinctes.\\
		
			\textbf{Deuxième cas :} deux valeurs propres égales parmi les trois.\\
		
			\textbf{Troisième cas :} \(Q(1) = Q(j) = Q(j^2)\).\\
		
		\end{enumerate}
	\end{solution}
	
	\section*{Exercice 71}
	\begin{statement}
	Soit \( P \) le plan d’équation \( x + y + z = 0 \) et \( D \) la droite d’équation \( x = \frac{y}{2} = \frac{z}{3} \).
	
	\begin{enumerate}
		\item Vérifier que \( \mathbb{R}^3 = P \oplus D \).
		\item Soit \( p \) la projection vectorielle de \( \mathbb{R}^3 \) sur \( P \) parallèlement à \( D \).\\
		Soit \( u = (x, y, z) \in \mathbb{R}^3 \).\\
		Déterminer \( p(u) \) et donner la matrice de \( p \) dans la base canonique de \( \mathbb{R}^3 \).
		\item Déterminer une base de \( \mathbb{R}^3 \) dans laquelle la matrice de \( p \) est diagonale.
	\end{enumerate}
	\end{statement}
	\begin{solution}
	\begin{enumerate}
		\item \( 1 + 2 + 3 \neq 0 \).\\
		\( \dim D + \dim P = 1 + 2 = 3 = \dim \mathbb{R}^3 \). 
		\item Par définition \( u - p(u) \in D \).\\
		Donc il existe \( \alpha \in \mathbb{R} \) tel que :
		\[
		u - p(u) = \alpha (1,2,3) \Rightarrow p(u) = (x - \alpha,\, y - 2\alpha,\, z - 3\alpha). \tag
		\]
		Par définition \(p(u) \in D \).\\
		\[
		x + y + z - 6\alpha = 0 \Rightarrow \alpha = \frac{1}{6}(x + y + z).
		\]
		\[
		p(u) = \frac{1}{6}(5x - y - z,\ -2x + 4y - 2z,\ -3x - 3y + 3z).
		\]
		\[
		A = \frac{1}{6} \begin{pmatrix}
			5 & -1 & -1 \\
			-2 & 4 & -2 \\
			-3 & -3 & 3
		\end{pmatrix}.
		\]
		
		\item:
		\[
		e_1' = (1,2,3),\quad e_2' = (1,-1,0),\quad e_3' = (0,1,-1).
		\]
	\end{enumerate}
	\end{solution}
	
\section*{Exercice 72}
\begin{statement}
	Soit \( n \) un entier naturel non nul.\\
	Soit \( f \) un endomorphisme d’un espace vectoriel \( E \) de dimension \( n \), et soit \( e = (e_1, \dots, e_n) \) une base de \( E \).\\
	On suppose que \( f(e_1) = f(e_2) = \dots = f(e_n) = v \), où \( v \) est un vecteur donné de \( E \).
	
	\begin{enumerate}
		\item Donner le rang de \( f \).
		\item \( f \) est-il diagonalisable ? (Discuter en fonction du vecteur \( v \).)
	\end{enumerate}
\end{statement}
\begin{solution}
	\begin{enumerate}
		\item 
		\item< Si \( f \) non nul
		\( \chi_f(X) = X^{n - 1}(X - \lambda) \) avec \( \lambda \neq 0 \). 
		\textbf{Premier sous-cas :} \( \lambda \neq 0 \)\\
		\textbf{Deuxième sous-cas :} \( \lambda = 0 \)\\
	\end{enumerate}
\end{solution}

	\section*{Exercice 73 \quad algèbre}
	\begin{statement}
		On pose \( A = \begin{pmatrix} 2 & 1 \\ 4 & -1 \end{pmatrix} \).
		\begin{enumerate}
			\item Déterminer les valeurs propres et les vecteurs propres de \( A \).
			\item Déterminer toutes les matrices qui commutent avec la matrice \( \begin{pmatrix} 3 & 0 \\ 0 & -2 \end{pmatrix} \).\\
			En déduire que l’ensemble des matrices qui commutent avec \( A \) est \( \mathrm{Vect}(I_2, A) \).
		\end{enumerate}
	\end{statement}
	
	\begin{solution}
		\begin{enumerate}
			\item \( \chi_A = (X - 3)(X + 2) \), donc \( \text{Sp}(A) = \{-2, 3\} \).\\
			\( AX = 3X \) et \( AX = -2X \) $\Rightarrow$ 
			
			\[
			E_3 = \mathrm{Vect}\left( \begin{pmatrix} 1 \\ 1 \end{pmatrix} \right)
			\quad \text{et} \quad
			E_{-2} = \mathrm{Vect}\left( \begin{pmatrix} 1 \\ -4 \end{pmatrix} \right).
			\]
			
			\item Soit \( N = \begin{pmatrix} a & b \\ c & d \end{pmatrix} \).\\
			\( D = \begin{pmatrix} 3 & 0 \\ 0 & -2 \end{pmatrix} \)
			\( ND = DN \) $\Rightarrow$ \\
			\[
			\begin{cases}
				-2b = 3b \\
				3c = -2c
			\end{cases}
			\]
			\( A = PDP^{-1} \)
			\[
			P = \begin{pmatrix} 1 & 1 \\ 1 & -4 \end{pmatrix}, \quad D = \begin{pmatrix} 3 & 0 \\ 0 & -2 \end{pmatrix}.
			\]
			\( AM = MA \iff P^{-1}MP = \begin{pmatrix} a & 0 \\ 0 & d \end{pmatrix} \iff M = P \begin{pmatrix} a & 0 \\ 0 & d \end{pmatrix} P^{-1} \).
		\end{enumerate}
	\end{solution}

	\section*{Exercice 74 \quad algèbre}
	\begin{statement}
		\begin{enumerate}
			\item On considère la matrice \( A = \begin{pmatrix} 1 & 0 & 2 \\ 0 & 1 & 0 \\ 2 & 0 & 1 \end{pmatrix} \).
			\begin{enumerate}
				\item Justifier sans calcul que \( A \) est diagonalisable.
				\item Déterminer les valeurs propres de \( A \) puis une base de vecteurs propres associés.
			\end{enumerate}
			\item On considère le système différentiel :
			\[
			\begin{cases}
				x' = x + 2z \\
				y' = y \\
				z' = 2x + z
			\end{cases}
			\qquad x, y, z \text{ désignant trois fonctions de la variable } t,
			\]
			dérivables sur \( \mathbb{R} \).\\
			En utilisant la question 1. et en le justifiant, résoudre ce système.
		\end{enumerate}
	\end{statement}
\begin{solution}
	\begin{enumerate}
		\item
		\[
		\chi_A(\lambda) = (\lambda - 1)(\lambda + 1)(\lambda - 3)
		\]
		\[
		E_1 = \mathrm{Vect}\left( \begin{pmatrix} 0 \\ 1 \\ 0 \end{pmatrix} \right), \quad
		E_{-1} = \mathrm{Vect}\left( \begin{pmatrix} 1 \\ 0 \\ -1 \end{pmatrix} \right), \quad
		E_3 = \mathrm{Vect}\left( \begin{pmatrix} 1 \\ 0 \\ 1 \end{pmatrix} \right)
		\]
		
		On pose donc la base propre :
		\[
		e'_1 = (0,1,0),\quad e'_2 = (1,0,-1),\quad e'_3 = (1,0,1)
		\]
		
		La base \( e' = (e'_1, e'_2, e'_3) \) est une base de vecteurs propres de \( A \).
		
		\item
		\[
		X'(t) = AX(t) \iff P^{-1}X' = DP^{-1}X
		\]
		
		On pose :
		\[
		X_1(t) = \begin{pmatrix} x_1(t) \\ y_1(t) \\ z_1(t) \end{pmatrix} = P^{-1}X(t)
		\]
		
		On résout :
		\[
		\begin{cases}
			x_1(t) = ae^t \\
			y_1(t) = be^{-t} \\
			z_1(t) = ce^{3t}
		\end{cases}
		\quad \text{avec } (a,b,c) \in \mathbb{R}^3
		\]
		
		On remonte à \( X(t) = P X_1(t) \), ce qui donne :
		\[
		\boxed{
			\begin{cases}
				x(t) = be^{-t} + ce^{3t} \\
				y(t) = ae^t \\
				z(t) = -be^{-t} + ce^{3t}
			\end{cases}
			\quad \text{avec } (a,b,c) \in \mathbb{R}^3
		}
		\]
		
	\end{enumerate}
\end{solution}

\section*{Exercice 75 \quad algèbre}
\begin{statement}
	On considère la matrice \( A = \begin{pmatrix} -1 & -4 \\ 1 & 3 \end{pmatrix} \).
	\begin{enumerate}
		\item Démontrer que \( A \) n’est pas diagonalisable.
		\item On note \( f \) l’endomorphisme de \( \mathbb{R}^2 \) canoniquement associé à \( A \).\\
		Trouver une base \( (v_1, v_2) \) de \( \mathbb{R}^2 \) dans laquelle la matrice de \( f \) est de la forme
		\[
		\begin{pmatrix}
			a & b \\
			0 & c
		\end{pmatrix}.
		\]
		On donnera explicitement les valeurs de \( a, b, c \).
		\item En déduire la résolution du système différentiel
		\[
		\begin{cases}
			x' = -x - 4y \\
			y' = x + 3y
		\end{cases}
		\]
	\end{enumerate}
\end{statement}

\begin{solution}
	\begin{enumerate}
		\item \[
		\chi_A(X) = (X - 1)^2 \Rightarrow \text{Sp}A = \{1\}
		\]
		\item \[
		E_1(A) = \mathrm{Vect} \left( \begin{pmatrix} 2 \\ -1 \end{pmatrix} \right)
		\]
		On choisit:
		\[
		v_1 = (2, -1), \quad v_2 = (-1, 0)
		\]
		la matrice de \( f \) 
		\[
		T = \begin{pmatrix} 1 & 1 \\ 0 & 1 \end{pmatrix}
		\]
		et la matrice de passage est :
		\[
		P = \begin{pmatrix} 2 & -1 \\ -1 & 0 \end{pmatrix}, \quad A = P T P^{-1}
		\]
		
		\item \[
		X' = A X \iff Y' = T Y
		\]
		ou :
		\[
		\begin{cases}
			a'(t) = a(t) + b(t) \\
			b'(t) = b(t)
		\end{cases}
		\]
		De solution générale :
		\[
		\begin{cases}
			b(t) = \mu e^{t} \\
			a(t) = \lambda e^{t} + \mu t e^{t}
		\end{cases}
		\quad \text{avec } (\lambda, \mu) \in \mathbb{R}^2
		\]
		\( X = P Y \) donne :
		\[
		\begin{cases}
			x(t) = ((2\lambda - \mu) + 2\mu t)e^{t} \\
			y(t) = (-\lambda + \mu t)e^{t}
		\end{cases}
		\quad \text{avec } (\lambda, \mu) \in \mathbb{R}^2
		\]
	\end{enumerate}
\end{solution}


	
	
	% Back matter
	\backmatter
	% Bibliography, glossary and index would go here
	
\end{document}